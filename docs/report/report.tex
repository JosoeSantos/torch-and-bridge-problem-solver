\documentclass[12pt,a4paper]{article}

% Pacotes básicos
\usepackage[utf8]{inputenc}    % Codificação do arquivo
\usepackage[T1]{fontenc}       % Acentos corretos
\usepackage[brazil]{babel}     % Português do Brasil
\usepackage{graphicx}          % Inclusão de imagens
\usepackage{float}             % Melhor controle de posição de figuras/tabelas
\usepackage{amsmath, amssymb}  % Símbolos matemáticos
\usepackage{hyperref}          % Links clicáveis
\usepackage{caption}           % Legendas personalizadas
\usepackage{cite}              % Gerenciamento de citações

\title{Relatório do Projeto de Grafos e Algoritmos de Busca}
\author{Grupo X \\ Disciplina Y \\ Universidade Z}
\date{\today}

\begin{document}

\maketitle
\tableofcontents
\newpage

\section{Introdução}
Nesta seção, deve ser apresentada uma visão geral do problema, os objetivos do relatório e uma breve descrição das tarefas realizadas.

\section{Modelagem do Problema em Forma de Grafo}
Descrever como o problema foi representado na forma de grafo.  
\begin{itemize}
    \item Definição dos vértices e arestas;
    \item Justificativa da modelagem escolhida;
    \item Exemplos ilustrativos com diagramas.
\end{itemize}

\section{Rotinas Desenvolvidas}
Explicar detalhadamente as rotinas implementadas.  
\begin{itemize}
    \item Estrutura geral do código;
    \item Funções principais;
    \item Decisões de implementação.
\end{itemize}

\section{Algoritmos de Busca Estudados}
Apresentar uma explicação básica dos algoritmos analisados.  
\begin{itemize}
    \item Busca em Largura (BFS);
    \item Busca em Profundidade (DFS);
    \item Outros algoritmos relevantes (caso aplicável).
\end{itemize}

\section{Comparação e Discussão dos Resultados}
Comparar os algoritmos aplicados para solucionar o puzzle.  
\begin{itemize}
    \item Critérios de comparação (tempo de execução, número de passos, consumo de memória, etc.);
    \item Tabelas e gráficos com resultados;
    \item Discussão dos pontos fortes e fracos de cada abordagem.
\end{itemize}

\section{Conclusão}
Apresentar as conclusões gerais do trabalho, destacando os principais aprendizados e possíveis melhorias futuras.

\section{Referências}
Inserir todas as referências utilizadas no mesmo formato (ABNT, APA ou Vancouver).  
Exemplo em ABNT:
\begin{itemize}
    \item CORMEN, T. H.; LEISERSON, C. E.; RIVEST, R. L.; STEIN, C. \textit{Algoritmos: Teoria e Prática}. 3. ed. Rio de Janeiro: Elsevier, 2012.
    \item KNUTH, D. E. \textit{The Art of Computer Programming}. Addison-Wesley, 1997.
\end{itemize}

\end{document}
